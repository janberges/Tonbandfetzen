\documentclass[a4paper, 10pt]{article}
\usepackage[utf8]{inputenc}
\usepackage[english]{babel}
\usepackage{amsmath, amsfonts, amssymb}
\usepackage{here}

\title{\itshape \ttfamily Tonbandfetzen}
\author{Jan Berges}
\date{March 29, 2015}

\begin{document}
	\maketitle
	
	\section*{Outline}
	
	The perl module {\it Tonbandfetzen} helps create audio data from plain text input, modify and arrange it and handle the {\it Audio Interchange File Format}.
	
	\section*{Global variables}
	
	There are several global variables ($\rightarrow$ table \ref{variables}) reserved for the information which is necessary to make the module collaborate with the main script.
	\begin{table}[H]
		\centering
		\begin{tabular}
		    { r            c              c                  l }
			{\bf variable} & {\bf value}  & {\bf initial --} & {\bf meaning}            \\
			\verb|$B|      & $8, 16, 32$  & 16               & {\it sampleSize}         \\
			\verb|$C|      & $1, 2$       & 2                & {\it numChannels}        \\
			\verb|$s|      &              & 44100            & second/{\it sampleRate}  \\
			\verb|$T|      & $> 0$        & \verb|$s|/2      & beat duration            \\
			\verb|$A4|     &              & 440/\verb|$s|    & standard pitch           \\
			\verb|$N|      & $1, 2\hdots$ & 12               & notes per octave         \\
			\verb|$pi|     & $\pi$        & $\pi$            & $\pi$                    \\
			\verb|@p|      & (mono        & (cubic           & single wave              \\
			\verb|@a|      & audio        & curves)          & attack envelope          \\
			\verb|@z|      & data)        &                  & reverse release envelope \\
		\end{tabular}
		\caption{global variables}
		\label{variables}
	\end{table}
	
	\section*{Audio data}
	
	Audio data is stored in one-dimensional arrays which are sorted chronologically and subordinately by channel according to the {\it Audio Interchange File Format}. Except for \verb|@p|, \verb|@a| and \verb|@z|, data transfer works via references.
	
	\section*{Subroutines}
	
	\subsection*{make}
	
	\begin{verbatim}
		make 'name.aif', fix $data
	\end{verbatim}
	\verb|make| saves data with respect to \verb|$B|, \verb|$C| and \verb|$s|.
	
	\subsection*{take}
	
	\begin{verbatim}
		$data = take 'name.aif'
	\end{verbatim}
	\verb|take| reads data and sets \verb|$B|, \verb|$C| and \verb|$s| accordingly.
	
	\subsection*{mel}
	
	\begin{verbatim}
		$theme = mel 'F#2 1 = 1 = 1 -4 3:4 +3 1:4 + 1 -4 3:4 +3 1:4 + 2'
	\end{verbatim}
	\verb|mel| creates audio data from a textual command sequence using the samples \verb|@p|, \verb|@a| and \verb|@z| and with respect to \verb|$C|, \verb|$s|, \verb|$T|, \verb|$A4| and \verb|$N|. The commands are separated by white space and usually feature a special sign ($\rightarrow$ table \ref{signs}), that not only defines the meaning but also the mathematical sign of the accompanying number, which may contain the decimal point and one dividing colon and be omitted if zero. The processing of a defined tone is initiated by the duration command, which lacks a sign, unless it indicates a pause. Initial and per \verb|$T| values hold until overwritten. The frequency may also be declared via scientific pitch notation, e.g. \verb|B3|, \verb|C4| or \verb|C#4|. If it is zero, \verb|@p| is not scaled temporally.
	\begin{table}[H]
		\centering
		\begin{tabular}
			{ l            l                                    l                c               c               c }
			{\bf quantity} & {\bf formula}                      & {\bf unit}     & {\bf initial} & {\bf per $t$} & {\bf per \verb|$T|} \\
			duration       & $t$                                & \verb|$T|      & (\verb|*|)    &               &                     \\
			frequency      & $f = f_0$                          & 1/\verb|$s|    &  \verb|~|     &               &                     \\
			none           &                                    &                &  \verb|=|     &               &                     \\
			interval       & $\operatorname{lb} \frac f {f_0}$  & oct./\verb|$N| & \verb|- +|    & \verb|\ /|    & \verb|_ ^|          \\
			level          & $\lg \frac{\sqrt{L^2 + R^2}}{y_0}$ & dB             & \verb|? !|    & \verb|> <|    & \verb|, ;|          \\
			-- difference  & $\lg \frac R L$                    & --             & \verb|[ ]|    & \verb|( )|    & \verb|{ }|
		\end{tabular}
		\caption{signs}
		\label{signs}
	\end{table}
	
	\subsection*{stack}
	
	\begin{verbatim}
		$band = stack $vocals, $guitar, $bass, $drums
	\end{verbatim}
	\verb|stack| superposes data, whereby shorter data is repeated.
	
	\subsection*{stick}
	
	\begin{verbatim}
		$song = stick $intro, $verse1, $chorus, $verse2, $chorus, $outro
	\end{verbatim}
	\verb|stick| juxtaposes data.
	
	\subsection*{fit}
	
	\verb|fit($a, $t)| returns data \verb|$a| scaled to duration \verb|$t| or \verb|$T| with respect to \verb|$C|.
	
	\subsection*{fix}
	
	\verb|fix($a, $y)| returns data \verb|$a| scaled to amplitude \verb|$y| or maximizes it for \verb|out| with respect to \verb|$B|. The original data is changed.
	
	\subsection*{cut}
	
	\verb|cut($a)| returns data \verb|$a| without leading or trailing zero blocks of length \verb|$C|.
	
	\subsection*{in}
	
	\begin{verbatim}
		@p = map { sin($_)      } in 0, $pi * 2;
		@a = map { sin($_) ** 2 } in 0, $pi / 2, $s / 20;
		@z = @a
	\end{verbatim}
	\verb|in($a, $b, $n)| returns an array of \verb|$n| or \verb|$s| equidistant numbers from \verb|$a| exclusively to \verb|$b| inclusively.
	
	\subsection*{noise}	
	
	\verb|noise($f, $i, $n, $t)| returns noise data of duration \verb|$t| or \verb|$s| that contains \verb|$n| or a hundred frequencies lying inside a range over \verb|$i| half tones or one octave centered geometrically around a given \verb|$f| or the standard pitch.
	
	\subsection*{ext/rex}
	
	\verb|ext($n)| returns the {\it 80 bit IEEE Standard 754} representation of the number \verb|$n|, \verb|rex| vice versa.
\end{document}
